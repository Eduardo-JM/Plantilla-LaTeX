\documentclass[12pt]{report}

% Se incluye el preámbulo, archivo donde se encuentran
% los paquetes, la configuración y los comandos definidos
%%%%%%%%%%%%%%%%%%%%%%%%%%%%%%%
% Configuración del documento

\usepackage[english,spanish,es-tabla]{babel}
\usepackage[
	letterpaper,
	top=18mm,
	bottom=20mm,
	right=20mm,
	left=25mm
]{geometry}
% En caso de no querer usar setlenght
%\usepackage[parfill]{parskip}
\usepackage{multicol}

\usepackage{ragged2e}
\usepackage{fancyhdr}
\usepackage{lastpage}
%%%%%%%%%%%%%%%%%%%%%%%%%%%%%%%

%%%%%%%%%%%%%%%%%%%%%%%%%%%%%%%
% Fuentes, codicación

\usepackage{titlesec}
\usepackage{fontspec}
\usepackage{xunicode}
\usepackage{xltxtra}
%%%%%%%%%%%%%%%%%%%%%%%%%%%%%%%

%%%%%%%%%%%%%%%%%%%%%%%%%%%%%%%
% Matemáticas y química

\usepackage{amsmath}
\usepackage{amsfonts}
\usepackage{siunitx}
\usepackage{xfrac}

\usepackage[
	modules={reactions,formula,redox,charges}
]{chemmacros}[2022/03/11]
%%%%%%%%%%%%%%%%%%%%%%%%%%%%%%%

%%%%%%%%%%%%%%%%%%%%%%%%%%%%%%%
% Hipervínculos, referencias
% y glosarios

\usepackage[hidelinks]{hyperref}
\usepackage[spanish]{cleveref}

\usepackage[
	translate=true,
	toc,
	acronym
]{glossaries}
\usepackage{glossary-mcols}

\usepackage[
	backend=biber,
	style=ieee,
%	Para apa usar
%	style=apa,
%	citestyle=apa,
	hyperref=true,
	url=false
]{biblatex}

\usepackage[thresholdtype=words]{csquotes}

%%%%%%%%%%%%%%%%%%%%%%%%%%%%%%%

%%%%%%%%%%%%%%%%%%%%%%%%%%%%%%%
% Tipografía y colores

\usepackage{xcolor}
\usepackage[shortlabels]{enumitem}
%%%%%%%%%%%%%%%%%%%%%%%%%%%%%%%

%%%%%%%%%%%%%%%%%%%%%%%%%%%%%%%
% Tablas y Figuras

\usepackage{graphicx}
\usepackage{caption}
\usepackage{subcaption}
\usepackage{wrapfig}

\usepackage{tabularray}
\UseTblrLibrary{varwidth}
%%%%%%%%%%%%%%%%%%%%%%%%%%%%%%%

%%%%%%%%%%%%%%%%%%%%%%%%%%%%%%%
% Dibujos y diagramas

\usepackage{tikz}
% Figuras geométricas, flechas, sombreado, cálculos y patrones
\usetikzlibrary{
	shapes.geometric,
	arrows.meta,
	shadows,
	calc,
	patterns,
	patterns.meta
}
% Usada para unos dibujos, quitar si quieres
\usepackage{chemplants}

% Cajas de colores
\usepackage[most]{tcolorbox}
\tcbuselibrary{raster}
\tcbuselibrary{external}
% NUNCA COMBINAR BREAKABLE Y EXTERNAL
\tcbuselibrary{breakable}
\tcbset{external/prefix=boxes/}
\tcbEXTERNALIZE

% Externalización
% Se necesita compilar con la bandera --shell-escape
\usetikzlibrary{external}

% Árboles (se usa para los diagramas)
\usepackage[edges, external]{forest}
\tikzexternalize[prefix=pictures/]
%%%%%%%%%%%%%%%%%%%%%%%%%%%%%%%

%%%%%%%%%%%%%%%%%%%%%%%%%%%%%%%
% Utilidades

\usepackage{adjustbox}
\usepackage{pdfpages}
%%%%%%%%%%%%%%%%%%%%%%%%%%%%%%%


%%%%%%%%%%%%%%%%%%%%%%%%%%%%%%%
% Texto y tipografía

% Vea CTAN parskip si prefiere eso
\setlength{\parindent}{0pt}
\setlength{\parskip}{0.5em}

\defaultfontfeatures{Ligatures=TeX}
\setmainfont{Times New Roman}

\titleformat{\chapter}[display]
    {\normalfont\huge\bfseries}
    {\chaptertitlename\ \thechapter}
    {20pt}{\Huge}
\titlespacing*{\chapter}{0pt}{0pt}{0pt}
%%%%%%%%%%%%%%%%%%%%%%%%%%%%%%%

%%%%%%%%%%%%%%%%%%%%%%%%%%%%%%%
% Lenguaje y localización

\selectlanguage{spanish}

\crefname{table}{\spanishtablename}{\spanishtablename}

\crefname{figure}{fig.}{figs.}
\Crefname{figure}{Fig.}{Figs.}

\crefname{reaction}{reacción}{reacciones}
\Crefname{Reaction}{Reacción}{Reacciones}

\crefname{section}{sección}{secciones}
\Crefname{Section}{Sección}{Secciones}

\decimalpoint
%%%%%%%%%%%%%%%%%%%%%%%%%%%%%%%

%%%%%%%%%%%%%%%%%%%%%%%%%%%%%%%
% Imágenes

\graphicspath{{./figures}}
%%%%%%%%%%%%%%%%%%%%%%%%%%%%%%%

%%%%%%%%%%%%%%%%%%%%%%%%%%%%%%%
% Tablas

\DefTblrTemplate{contfoot-text}{normal}{\scriptsize Continua en la siguiente página}
\SetTblrTemplate{contfoot-text}{normal}
\DefTblrTemplate{conthead-text}{normal}{(Continuación)}
\SetTblrTemplate{conthead-text}{normal}
%%%%%%%%%%%%%%%%%%%%%%%%%%%%%%%

%%%%%%%%%%%%%%%%%%%%%%%%%%%%%%%
% Colores del documento

\definecolor{primary}{HTML}{082D70}
\definecolor{primary-var}{HTML}{3059A8}
\definecolor{secondary}{HTML}{086070}
\definecolor{secondary-var}{HTML}{35BAE5}
%\definecolor{primary-accent}{HTML}{000000}
%\definecolor{secondary-accent}{HTML}{000000}
\definecolor{background}{HTML}{E7EBF4}

% Color del politécnico
\definecolor{cherry}{RGB}{90, 18, 54}

%% Cover page
\definecolor{ultralightgreen}{RGB}{244, 249, 241}
\definecolor{green}{RGB}{146, 208, 80}


% Other colors
\definecolor{urlColor}{RGB}{89,156,255}

% Other colors
\definecolor{lightblue}{RGB}{176, 221, 255}
\definecolor{lilac}{RGB}{174, 182, 211}
\definecolor{lightgreen}{RGB}{150, 240, 180}
%%%%%%%%%%%%%%%%%%%%%%%%%%%%%%%

%%%%%%%%%%%%%%%%%%%%%%%%%%%%%%%
% Matemáticas y química

\sisetup{
	per-mode=symbol,
	sticky-per,
	exponent-mode = engineering
}

%\DeclareSIUnit\day{day}

\chemsetup{
	reactions/own-counter = true
}
%%%%%%%%%%%%%%%%%%%%%%%%%%%%%%%

%%%%%%%%%%%%%%%%%%%%%%%%%%%%%%%
% Hipervínculos, referencias
% y glosarios

\DeclareFieldFormat{url}{%
%  \mkbibacro{URL}\addcolon\space
	\href{#1}{%
		\nolinkurl{\thefield{urlraw}}%
	}%
}

\hypersetup{
	colorlinks = true,
	linkcolor = black,
	citecolor = primary,
	urlcolor = urlColor
}

\newglossarystyle{custommcolalttree}{%
	\setglossarystyle{mcolalttree}%
	\renewcommand{\glossaryheader}{\vspace{\dimexpr -\baselineskip -\parskip}}
}

\makeglossaries
\loadglsentries{bib/glossary.tex}
\loadglsentries{bib/acronym.tex}

\addbibresource{bib/References.bib}

\setquotestyle[mexican]{spanish}
\SetBlockThreshold{40}
%%%%%%%%%%%%%%%%%%%%%%%%%%%%%%%


%%%%%%%%%%%%%%%%%%%%%%%%%%%%%%%%%%%%%
% Headers & Footers

\fancypagestyle{generalfancy}{%
	\fancyhf{}
	\rfoot{\thepage}
}

\fancypagestyle{fancycover}{%
	\fancyhead[C]{\makeipnheader}
	\renewcommand{\footrulewidth}{0.5pt}
	\fancyfoot[L]{%
		\noindent\justifying\tiny\textcolor{gray!80}{%
			\textbf{AVISO DE PRIVACIDAD}: Los datos recabados serán protegidos, incorporados y tratados por el Departamento de Tecnologías Avanzadas de la UPIITA, cuya finalidad es el uso de los mismos exclusivamente para el proceso de registro de protocolos y proyectos de trabajo terminal y proyecto terminal, para la titulación por opción curricular de las carreras de Ingeniería Biónica, Ingeniería Mecatrónica, Ingeniería Telemática e Ingeniería en Sistemas Automotrices. El responsable de los datos personales es el Departamento de Tecnologías Avanzadas, perteneciente a la Subdirección Académica de la Unidad Profesional Interdisciplinaria en Ingeniería y Tecnologías Avanzadas del IPN, donde el interesado podrá ejercer los derechos de acceso y corrección en la dirección: Av. Instituto Politécnico Nacional No. Col. Barrio la Laguna Ticomán, Gustavo A. Madero, México DF, CP. 07340, en el Edificio 1 primer piso. Lo anterior se informa en cumplimiento del Decimoséptimo de los Lineamientos de Protección de Datos Personales, publicados en el Diario Oficial de la Federación el 30 de septiembre de 2005.%
		}%
	}%
}

\pagestyle{fancy}
\fancyhf{}
\rfoot{\thepage}
\renewcommand{\headrulewidth}{0pt}
%%%%%%%%%%%%%%%%%%%%%%%%%%%%%%%%%%%%%
%%%%%%%%%%%%%%%%%%%%%%%%%%%%%%%
% Tipografía y color

%Listas

\SetEnumitemKey{columns}{
	before = \begin{multicols}{#1},
	after = \end{multicols}
}

%%%%%%%%%%%%%%%%%%%%%%%%%%%%%%%

%%%%%%%%%%%%%%%%%%%%%%%%%%%%%%%
% Dibujos y diagramas

\pgfkeys{/forest,
	rect/.append style = {
		rectangle,
		rounded corners=5pt,
		/tikz/align=center,
		inner color=background,
		drop shadow,
	},
	circ/.append style = {circle, /tikz/align=center}	
}

%%%%%%%%%%%%%%%%%%%%%%%%%%%%%%%

%%%%%%%%%%%%%%%%%%%%%%%%%%%%%%%
% Forest curly bracket edges

\forestset{
    declare dimen = {curly bracket sep}{0.5em},
    curly bracket edge’/.style={
        edge={rotate/.option=!parent.grow},
        edge path'= {
			[
				color=linecol,
				rounded corners=5pt,
				>={Stealth[length=8pt]},
				line width=0.5pt,
				->
			]
				% !u. means up (upper parent)
				(!u.parent anchor) -- ++(\forestoption{curly bracket sep},0) |- (.child anchor)
		},
    },
    curly bracket edge/.style={
        on invalid={fake}{!parent.parent anchor=children},
        child anchor=parent,
        curly bracket edge’
    },
    curly bracket edges/.style={for nodewalk={#1}{curly bracket edge}},
    curly bracket edges/.default=tree,
}
%%%%%%%%%%%%%%%%%%%%%%%%%%%%%%%

%%%%%%%%%%%%%%%%%%%%%%%%%%%%%%%
% Forest arrowed folder edges

\forestset{
	declare dimen register=arrowed folder indent,
	arrowed folder indent=0.45em,
	arrowed folder/.style={
		parent anchor=-children last,
		anchor=parent first,
		calign=child,
		calign primary child=1,
		for children={
			child anchor=parent,
			anchor=parent first,
			edge={rotate/.option=!parent.grow},
			edge path'/.expanded={
				[
					color=linecol,
					rounded corners=2pt,
					>={Stealth[length=6pt]},
					line width=0.5pt,
					->
				]
				([xshift=\forestregister{arrowed folder indent}]!u.parent anchor) |- (.child anchor)
			},
		},
		after packing node={
			if n children=0{}{
				tempdiml=l_sep()-1*l("!1"),
				tempdims={-abs(max_s("","")-min_s("",""))-s_sep()},
				for children={
					l+=tempdiml,
					s+=tempdims()*(reversed()-0.5)*2,
				},
			},
		},
	}
}
%%%%%%%%%%%%%%%%%%%%%%%%%%%%%%%
%%%%%%%%%%%%%%%%%%%%%%%%%%%%%%%%%%%%%
% Cover page
\makeatletter
	\newlength{\imagewidth}
	\newlength{\imageheight}
	\newlength{\imagesize}
	
	
	\setlength\imageheight{20mm}
	\setlength\imagewidth{12.5mm}
	\setlength\imagesize{21.9mm}
	
	\newcommand{\makeipnheader}{
		\begin{minipage}[t]{\textwidth}
			\begin{minipage}{\imagesize}
				\includegraphics[
					width=\imagewidth,
					height=\imageheight,
					keepaspectratio
				]{logos/IPN.png}
				\newline
		    		\raggedright\footnotesize{DTA-PPT-01}
			\end{minipage}
			\hfill
			\begin{minipage}{\dimexpr\linewidth - 2\imagesize\relax - \imagewidth}
				\bgroup
					\scshape
        			\begin{center}            			
           				\fontsize{12pt}{12pt}\selectfont
           				\textbf{INSTITUTO POLITÉCNICO NACIONAL}\\
           				\fontsize{9pt}{9pt}\selectfont
           				UNIDAD PROFESIONAL INTERDISCIPLINARIA EN INGENIERÍA Y TECNOLOGÍAS AVANZADAS\\
           				\fontsize{10pt}{10pt}\selectfont
           				\textbf{PROTOCOLO}\\
           				\fontsize{14pt}{14pt}\selectfont
           				\textbf{INGENIERÍA EN ENERGÍA}	            		
        			\end{center}
       			\egroup
			\end{minipage}
			\hfill
			\begin{minipage}{\imagesize}
		        	\includegraphics[
		        		width=\imagesize,
		        		height=\imageheight,
		        		keepaspectratio
		        	]{logos/UPIITA.png}
		    	\end{minipage}
		\end{minipage}
	}
	
\makeatother

\makeatletter
	\newenvironment{coverpage}{%
		\newgeometry{
			top=10mm,
			left=20mm,
			right=20mm,
			headsep=0pt,
			headheight=27mm,
			footskip=7mm,
			includehead
		}
		\thispagestyle{fancycover}
	}
	{
		\vfill
		\newpage
		\restoregeometry
		\pagestyle{generalfancy}
	}
\makeatother
%%%%%%%%%%%%%%%%%%%%%%%%%%%%%%%%%%%%%

%%%%%%%%%%%%%%%%%%%%%%%%%%%%%%%%%%%%%
% Math conditions

% Usage
%\begin{conditions} key & description \end{conditions}
\newenvironment{conditions}
  {\par\vspace{\abovedisplayskip}\noindent\begin{tabular}{>{$}l<{$} @{${}:{}$} l}}
  {\end{tabular}\par\vspace{\belowdisplayskip}}

% Usage
%\begin{conditions*} key & symbol & description \end{conditions*}

\newenvironment{conditions*}
  {\par\vspace{\abovedisplayskip}\noindent
   \tabularx{\columnwidth}{>{$}l<{$} @{}>{${}}c<{{}$}@{} >{\raggedright\arraybackslash}X}}
  {\endtabularx\par\vspace{\belowdisplayskip}}
%%%%%%%%%%%%%%%%%%%%%%%%%%%%%%%%%%%%%

%%%%%%%%%%%%%%%%%%%%%%%%%%%%%%%%%%%%%
% abstract
\makeatletter
	\renewenvironment{abstract}{%
		\if@twocolumn
			\chapter*{\abstractname}%
		\else %% <- here I've removed \small
			\begin{center}%
				{\normalfont\fontsize{16}{15}\bfseries\abstractname\vspace{\z@}}
			\end{center}%
		\quotation
		\fi
	}
	{
		\if@twocolumn\else\endquotation\fi
	}
\makeatother
%%%%%%%%%%%%%%%%%%%%%%%%%%%%%%%%%%%%%

%%%%%%%%%%%%%%%%%%%%%%%%%%%%%%%%%%%%%
% Table

% tabularray permite envolver el contenido de una
% celda en un comando o environment. Este comando
% se usa para celdas que incluyen tikz como parte de
% su contenido. Véase la metodología para un ejemplo de uso

% To create cells wrapped by commands
\NewDocumentCommand{\adjusttikzpic}{m}{%
  \adjustbox{valign=m}{%
  	\inputtikz{#1}
  }
}
%%%%%%%%%%%%%%%%%%%%%%%%%%%%%%%%%%%%%

%%%%%%%%%%%%%%%%%%%%%%%%%%%%%%%
% Comandos del documento

\makeatletter                       
	\def\printauthor{\@author}
	\def\printdate{\@date}
	\def\printtitle{\@title}
\makeatother

\providecommand{\keywords}[2][Palabras clave]{
  \small	
  \textbf{\textit{#1 ---}} #2
}
% Comando para incluir figuras tikz
% El argumento recibido es el archivo a dibujar
\NewDocumentCommand{\inputtikz}{m}{
	\tikzsetnextfilename{#1}
	\input{tikz/#1.tex}
}

% Comando para incluir diagramas tikz (forest)
% El argumento recibido es el archivo a dibujar
\NewDocumentCommand{\inputdiagram}{m}{
	\input{diagrams/#1.tex}
}

% Comando para incluir cajas de colores (tcolorbox)
% El argumento recibido es la caja a crear
\NewDocumentCommand{\inputtcb}{m}{%
	\input{tcolorbox/#1.tex}%
}
%%%%%%%%%%%%%%%%%%%%%%%%%%%%%%%

%%%%%%%%%%%%%%%%%%%%%%%%%%%%%%%
% Datos del documento

\title{TÍTULO DE TU PROTOCOLO DE INVESTIGACIÓN, MENOR A CIERTA EXTENSIÓN}
\date{fecha de entrega}
\author{Autor del documento}

% Para la fecha de hoy
%\date{\today}
%%%%%%%%%%%%%%%%%%%%%%%%%%%%%%%
						

\begin{document}
	% Se detiene la numeración
	\pagenumbering{gobble}
	
	% Se inicia la portada
	\begin{coverpage}
		% Se incluye el archivo de la portada
		%\thispagestyle{fancycover}
\bgroup
	\fontsize{9pt}{9pt}\selectfont
	\begin{tblr}{
		width=\linewidth,
		colspec = { *{12}X },
		hlines,
		vlines,
		stretch=1.5,
		row{1} = {
			bg = green,
			font =  \bfseries
		},
		rows = {m}
	}
		% Título del protocolo %
		\SetCell[c=12]{c}
		TÍTULO DEL PROTOCOLO &&&&&&&&&&&\\ %
		%
		\SetCell[c=12]{l,\linewidth}
		\textbf{\printtitle}
		&&&&&&&&&&& \\ 
		% Datos del protocolo%
		\SetRow{green}
		\SetCell[c=12]{c}
			\textbf{DATOS DEL PROTOCOLO} &&&&&&&&&&&
		\\ 
		\SetCell[c=5]{r}
			{
			\textbf{Número de revisión}\\
			\scriptsize{(Primera, segunda tercera o Protocolo para registro)}
			}
		&&&&&
		\SetCell[c=2]{c}
			\textbf{XXXXXXX}
		&&
		\SetCell[c=3]{r}
			\textbf{Semestre} &&&
		\SetCell[c=2]{c}
			\textbf{2023-1} 
		&\\ 
		\SetCell[c=5]{r}
			{
			\textbf{Número de proyecto asignado} \\
			\scriptsize{(Número asignado por el profesor de especialidad)}
			}
		&&&&&
		\SetCell[c=2]{c}
			\centering \textbf{1}
		&&
		\SetCell[c=3]{r}
			{
			\textbf{Fecha}\\
			\scriptsize{(Fecha programada)}
			}
		&&&
		\SetCell[c=2]{c}
			\centering \textbf{\printdate}
		&\\ 
		\SetCell[c=5]{r}
			{
			\textbf{Confidencialidad} \\
			\scriptsize{(Público o confidencial, incluir documento que lo avale)}
			}
		&&&&&
		\SetCell[c=2]{c}
			\centering \textbf{Público}
		&&
		\SetCell[c=3]{r}
			{
			\textbf{Número de Hojas}\\
			\scriptsize{(Cantidad de Hojas del Protocolo)}
			}
		&&&
		\SetCell[c=2]{c}
			\centering \textbf{\pageref{LastPage}}
		&\\ 
		\SetCell[c=7]{r}
			{
			\textbf{Patrocinador} \\
			\scriptsize{(En caso de existir, incluir el nombre en caso contrario dejar en blanco)}
			}
		&&&&&&&
		\SetCell[c=5]{c} { Modificar o quitar este texto } &&&& \\ 
		\SetCell[c=7]{r}
			{
			\textbf{Número Convenio o Registro} \\
			\scriptsize{(Incluir número de convenio patrocinio o número de proyecto de investigación que patrocina)}
			}
		&&&&&&&
		\SetCell[c=5]{c} { Modificar o quitar este texto } &&&&\\ 
	\end{tblr}
\egroup

\bgroup
	\fontsize{8pt}{8pt}\selectfont
	\begin{tblr}{
		width=\linewidth,
		colspec = { *{12}X },
		hlines,
		vlines,
		stretch=1.2,
		row{1} = {
			bg = green
		},
		rows = {m}
	}
		\SetCell[c=12]{c}
			\centerline{ALUMNO 1}
		&&&&&&&&&&&\\ 
		% Datos alumno %
		\SetRow{ultralightgreen}
		\SetCell[c=8]{c}
			DATOS ALUMNO 1
		&&&&&&&&
		\SetCell[c=4]{c}
			FIRMA
		&&&\\ 
		% Inicio datos%
		\SetCell[c=3]{r}
			Nombre del alumno
		&&&
		\SetCell[c=5]{l}
			Mi nombre va aquí
		&&&&&
		\SetCell[c=4,r=4]{c} ~&&&\\
		\SetCell[c=3]{r}
			Número de boleta
		&&&
		\SetCell[c=5]{l}
			\raggedright 2010640000
		&&&&&
		\SetCell[c=4]{c} &&&\\
		\SetCell[c=3]{r}
			Teléfono
		&&&
		\SetCell[c=5]{l}
			+52 1 5555 555 555
		&&&&&
		\SetCell[c=4]{c} ~&&&\\
		\SetCell[c=3]{r}
			Correo electrónico
		&&&
		\SetCell[c=5]{l}
			micorreoinstitucional@alumno.ipn.mx
		&&&&&
		\SetCell[c=4]{c} ~&&&\\ 
	\end{tblr}
	
	\begin{tblr}{
		width=\linewidth,
		colspec = { *{3}X X[1.5] X X[-2] X X[-3] *{4}X},
		hlines,
		vlines,
		row{1} = {
			bg = green
		},
		rows = {m},
	}
		% ASESOR %
		\SetCell[c=8]{c,0.666\linewidth - 2\tabcolsep}
			DATOS ASESOR 1
		&&&&&&&& 
		\SetCell[c=4]{c,0.333\linewidth - 2\tabcolsep}
			VISTO BUENO ASESOR 1
		&&&\\ 
		\SetCell[c=3]{r}
			Nombre Asesor \scriptsize{(Grado Académico)}
		&&&
		\SetCell[c=5]{r}
			El nombre del asesor va aquí
		&&&&&
		\SetCell[c=4,r=4]{c} ~ &&& \\
		%%%%
		\SetCell[c=3]{r}
			Academia
		&&&
		% En el segundo asesor borrar la footnotemark
		XXXXXXXXXXX	& Interno & X & {Externo\footnotemark[1]} & ~ &
		\SetCell[c=4]{c}
		~ &&&
		\\
		%%%%
		\SetCell[c=3]{r}
			Cédula profesional \scriptsize{(Obligatorio)}
		&&&
		\SetCell[c=5]{c}
			XXXXXXX
		&&&&&
		\SetCell[c=4]{c} ~ &&&\\
		%%%%
		\SetCell[c=3]{r}
			Correo electrónico
		&&&
		\SetCell[c=5]{c}
			correodelasesor@ipn.mx
		&&&&&
		\SetCell[c=4]{c} ~ &&&\\ 
	\end{tblr}
\egroup

\footnotetext[1]{\fontsize{8pt}{8pt}\selectfont En caso de Asesores Externos, deberá incluirse copia de su Cédula Profesional y Curriculum Vitae resumido en un archivo anexo al Protocolo.}
	\end{coverpage}
	\newpage
	
	% Índice, lista de figuras y lista de tablas
	\tableofcontents
	\listoffigures
	\listoftables
	
	\newpage
	% Abreviaciones y acrónimos
	\glsfindwidesttoplevelname[\acronymtype]
	\printglossary[
		style=custommcolalttree,
		type=\acronymtype,
		title=Abreviaciones y acrónimos,
		nonumberlist
	]
	
	% Glosario
	\glsfindwidesttoplevelname[\glsdefaulttype]
	\printglossary[style=alttree, nonumberlist]
	\newpage
	
	% Se reanuda la numeración arábiga
	\pagenumbering{arabic}
	\setcounter{page}{1}
	
	% Se incluyen los archivos del contenido
	\begin{abstract}
	\addcontentsline{toc}{chapter}{Resumen}
	
	\noindent Aquí va mi resumen. Usa noindent para evitar la sangría
	
	\keywords{lista, de, palabras, clave}
\end{abstract}

% Importante no borrar bgroup
\bgroup
	\selectlanguage{english}
	\begin{abstract}
		\noindent Here goes my abstract
		
		\keywords[Index terms]{keywords}
	\end{abstract}
\egroup	
	\chapter{Introducción}
	
	Redacta bien tu introducción con las especificaciones que te den en metodología, lleva algunas limitantes y debe responder ciertas preguntas, \href{http://scielo.sld.cu/scielo.php?script=sci_arttext&pid=S0864-21252010000200018}{Visita esta página}
	\chapter{Planteamiento del problema}

	{\Large \textbf{Síntesis de } \cite{itriago_c_planteamiento_2011}}
	
	\textbf{	Problema:}
	
	\begin{itemize}
		\item Estoy aquí ¿Cómo puedo llegar allá?
		\item Definir punto de partida, definir punto de llegada
		\item Necesidad actual
		\item Dato, meta, condiciones y operaciones
	\end{itemize}
	
	El planteamiento del problema constituye una demostración de cómo se está percibiendo la situación que se quiere investigar o resolver, de allí su importancia para la definición del Proyecto de Investigación.
	
	\begin{itemize}
		\item Puedo presentar en un par de párrafos el problema a tratar
		\item Indicar las restricciones
		\item Deriva de una necesidad de mejora
		\item Requieren herramientas y conocimiento
		\item ¿Opitimizas o inventas?
	\end{itemize}
	
	\textbf{INTERROGANTES}
	
	\begin{itemize}
		\item ¿Qué es lo que está sucediendo y cómo está sucediendo?
		\item ¿Desde cuándo?
		\item ¿Dónde está sucediendo?
		\item ¿Quién ha investigado antes sobre eso?
		\item ¿Por qué sucede?
		\item ¿Qué resulta problematizado para lo que se describe?
		\item ¿Para quién o quienes es un problema?
		\item ¿Qué puede hacerse para intentar resolver el problema?: se formula aquí claramente la meta o el propósito para la situación planteada, indicando de manera general lo que se puede hacer o bien la forma como se puede operar para superar exitosamente el problema que se está planteando
	\end{itemize}
	\chapter{Justificación}

	La justificación debe responder a varias preguntas importantes que sustentan la relevancia y pertinencia del proyecto.
	
	\begin{itemize}
		\item ¿Por qué realizar este proyecto?
		\item ¿Cómo aporta la ingeniería en energía al proyecto?
		\item ¿Por qué esta investigación es relevante?
		\item ¿Quiénes se benefician del desarrollo de este trabajo?
		\item ¿Qué aportes o contribuciones surgen de esta investigación?
	\end{itemize}
	
	\begin{figure}[!htb]
		\centering
		\includegraphics[width=\linewidth, height=70mm, keepaspectratio]{Delete/Justificación.png}
		\caption{No hagan esto}
		\label{fig:justificación}
	\end{figure}
	\chapter{Objetivos e hipótesis} \label{sec:objetivos}

	\section{Objetivo general}
		
		Comenzar con un verbo debo, no confundir verboides pues infundir errores
	
	\section{Objetivos específicos}
	
		\subsection{Trabajo terminal I}
		
			\begin{enumerate}
				\item Enumera tus objetivos
				\item Y divídelos por TT1
			\end{enumerate}
		
		\subsection{Trabajo terminal II}
			
			\begin{enumerate}
				\item Y TT2
				\item No seas muy ambicioso ni muy específico
				\item Date margen de cambio
			\end{enumerate}
			
\chapter{Hipótesis}
				
	Proposición o enunciado que afirma algo que aún no ha sido corroborado y a partir del cual se puede desarrollar una investigación con el objeto de verificar si la proposición era correcta.
	% NO borrar
\begingroup
\titleclass{\chapter}{straight}
\chapter{Antecedentes}

	Antedecentes: Cualquier trabajo asociado a mi proyecto con más de 5 años de ser publicado
	
	Aquí un acrónimo \acrfull{onu}. Aquí un término de glosario \gls{energia_termica}. Aquí una cita \cite{nadeeshani_nicotinamide_2022}

\endgroup
	\chapter{Estado del arte}
	
	Cualquier trabajo asociado a mi proyecto con 5 o menos años de ser publicado
	
	\chapter{Marco teórico}
	
	En esta parte aprendes mucho c:
	
	\inputdiagram{FirstDiagram}
	\chapter{Metodología de la investigación}

	Tengan mucho cuidado y describan bien lo que harán (A DETALLE)
	
	\inputtcb{Metas}
	
	\begin{longtblr}[
			caption = {Lista de posibles materiales identificados hasta el momento},
			label = {table:materiales_identificados}
		]{
			colspec = {c X},
			hlines,
			vlines,
			row{odd} = {bg=background},
			row{1} = {
				bg = cherry,
				fg=white,
				font =  \large\bfseries
			},
			rowhead = 1,
			column{1}={c},
			rows = {
				valign = m
			},
			cell{2-Z}{1} = {cmd=\adjusttikzpic},
		}
			Simbología & Descripción \\
			centrifugal-pump
				& \textbf{Bomba centrífuga}: Se usará para bombear agua\\
			check-valve
				& {\textbf{Válvula check}: Se usará para regular y mantener un flujo unidireccional}
		\end{longtblr}
	\chapter{Administración del proyecto}
	
	\section{Presupuesto estimado e infraestructura}
	
		\subsection{Recursos humanos}
			% Esto sólo es una propuesta
			\begin{longtblr}[
				caption = {Tabla de Recursos humanos},
				label = {table:recursos-humanos}
			]{
				colspec = {X l l c X},
				hlines,
				vlines,
				row{1} = {
					bg = cherry,
					fg=white,
					font = \bfseries
				},
				row{odd[2]} = {bg=background},
				rows={m}
			}
				Nombre & Papel & Grado académico & Institución & Tiempo de trabajo \\
				Tu nombre & Estudiante & Bachillerato técnico & escuela & 36 semanas \\
				Tu asesor & Asesor & Doctorado & escuela & Bajo disponibilidad
			\end{longtblr}
	
		\subsection{Recursos materiales}
			% Esto sólo es una propuesta
			\begin{longtblr}[
				caption = {Tabla de Recursos materiales},
				label = {table:recursos-materiales}
			]{
				colspec = {X c c r},
				hlines,
				vlines,
				row{1} = {
					bg = cherry,
					fg=white,
					font = \bfseries
				},
				row{odd[2]} = {bg=background},
				rows={m},
				cell{2-Z}{3,4} = {
					preto = {\$},%prepend to
				},
			}
				Insumos requeridos & Cantidad & Precio unitario & Total \\
				Arduino Uno R3 [A000066] & 1 & 499 & 499 \\
				Bomba De Agua De Diafragma \sfrac{1}{2} & 2 & 269 & 538\\
				Soldadura de estaño de \qty{1}{\mm} & 1 tubo & 87 & 87\\
				\SetCell[c=4]{r} {\bfseries Total: \$1124} &&&
			\end{longtblr}
	
		\subsection{Recursos técnicos}
			% Esto sólo es una propuesta
			\begin{longtblr}[
				caption = {Tabla de Recursos técnicos},
				label = {table:recursos-tecnicos}
			]{
				colspec = {X X[2] X[2] l l},
				hlines,
				vlines,
				measure=vbox,
				row{1} = {
					bg = cherry,
					fg=white,
					font = \bfseries
				},
				row{odd[2]} = {bg=background},
				rows={m},
				cell{2-Z}{5} = {
					preto = {\$},%prepend to
				},
			}
				Recurso & Descripción & Uso & Cantidad & Costo \\
				Git y Github 
					& Administrador de versiones
					& Se usarán en conjunto para tener un control adecuado de los cambios del proyecto
					& 1 Licencia y 1 cuenta
					& 0 \\
				Laptop o computador 
					& {
						\begin{itemize}[nosep]
							\item 4 GB Memoria RAM
							\item 12 GB Memoria interna libres
							\item Procesador \qty{2}{\giga\hertz}
							\item Sistema operativo:
								\begin{itemize}[nosep]
									\item Windows 8.1+
									\item macOS 10.14+
									\item Linux: Debian 9+, Ubuntu 16.04+, CentOS 7+, Fedora 30+, Alpine 3.13+, Otra distribución GLIBC 2.17+
								\end{itemize}
						\end{itemize}
					} 
					& La computadora o laptop se usará para el desarrollo de la programación requerida por el proyecto 
					& 1
					& 2800
				\\
				IntelliJ Idea community 
					& IDE para desarrollo de JVM y Android 
					& Se usará este software para facilitar y eficientar el desarrollo de software requerido
					& 1
					& 0
				\\
				\SetCell[c=5]{r} {\bfseries Total: \$2800}
			\end{longtblr}
	
	\section{Cronograma de actividades}
		\subsection{Trabajo terminal I}
			\begin{longtblr}[
				caption = {Tabla de actividades planeadas para TT1},
				label = {table:actividades-tt1}
			]{
				colspec = {X *{18}c},
				hlines,
				vlines,
				colsep=0.3mm,
				row{1} = {
					bg = cherry,
					fg=white,
					font =  \bfseries
				},
				row{odd[2]} = {bg=background},
				rowhead = 2,
				rows={m},
				cell{3-4}{2} = {bg=cherry},
				cell{5}{2-3} = {bg=cherry},
				cell{6}{3-4} = {bg=cherry},
				cell{7}{5-6} = {bg=cherry},
				cell{8}{5-7} = {bg=cherry},
				cell{9}{6-7} = {bg=cherry},
				cell{10}{7-8} = {bg=cherry},
				cell{11}{8-10} = {bg=cherry},
				cell{12}{11-12} = {bg=cherry},
				cell{13}{12-14} = {bg=cherry},
				cell{14}{14-16} = {bg=cherry},
				cell{15}{15-17} = {bg=cherry},
				cell{16}{18-19} = {bg=cherry},
				cell{17}{9-19} = {bg=cherry},
			}

				\SetCell[c=19]{c} Trabajo terminal I (semanas) &&&&&&&&&&&&&&&&&\\
				Actividad & 01 & 02 & 03 & 04 & 05 & 06 & 07 & 08 & 09 & 10 & 11 & 12 & 13 & 14 & 15 & 16 & 17 & 18 \\
				ACTIVIDAD 1 &&&&&&&&&&&&&&&&&&\\
				ACTIVIDAD 2 &&&&&&&&&&&&&&&&&&\\
				ACTIVIDAD 3 &&&&&&&&&&&&&&&&&&\\
				ACTIVIDAD 4 &&&&&&&&&&&&&&&&&&\\
				ACTIVIDAD 5 &&&&&&&&&&&&&&&&&&\\
				ACTIVIDAD 6 &&&&&&&&&&&&&&&&&&\\
				ACTIVIDAD 7 &&&&&&&&&&&&&&&&&&\\
				ACTIVIDAD 8 &&&&&&&&&&&&&&&&&&\\
				ACTIVIDAD 9 &&&&&&&&&&&&&&&&&&\\
				ACTIVIDAD 10 &&&&&&&&&&&&&&&&&&\\
				ACTIVIDAD 11 &&&&&&&&&&&&&&&&&&\\
				ACTIVIDAD 12 &&&&&&&&&&&&&&&&&&\\
				ACTIVIDAD 13 &&&&&&&&&&&&&&&&&&\\
				ACTIVIDAD 14 &&&&&&&&&&&&&&&&&&\\
				ACTIVIDAD 15 &&&&&&&&&&&&&&&&&&
			\end{longtblr}
			
		\subsection{Trabajo terminal II}
			
					
			\begin{longtblr}[
				caption = {Tabla de actividades planeadas para TT2},
				label = {table:actividades-tt2}
			]{
				colspec = {X *{18}c},
				hlines,
				vlines,
				colsep=0.3mm,
				row{1} = {
					bg = cherry,
					fg=white,
					font =  \large\bfseries
				},
				row{odd[2]} = {bg=background},
				rowhead = 2,
				rows={m},
				cell{3-4}{2} = {bg=cherry},
				cell{5}{2-3} = {bg=cherry},
				cell{6}{3-4} = {bg=cherry},
				cell{7}{5-6} = {bg=cherry},
				cell{8}{5-7} = {bg=cherry},
				cell{9}{6-7} = {bg=cherry},
				cell{10}{7-8} = {bg=cherry},
				cell{11}{8-10} = {bg=cherry},
				cell{12}{11-12} = {bg=cherry},
				cell{13}{12-14} = {bg=cherry},
				cell{14}{14-16} = {bg=cherry},
				cell{15}{15-17} = {bg=cherry},
				cell{16}{18-19} = {bg=cherry},
				cell{17}{9-19} = {bg=cherry},
			}

				\SetCell[c=19]{c} Trabajo terminal II (semanas) &&&&&&&&&&&&&&&&&\\
				Actividad & 01 & 02 & 03 & 04 & 05 & 06 & 07 & 08 & 09 & 10 & 11 & 12 & 13 & 14 & 15 & 16 & 17 & 18 \\
				ACTIVIDAD 1 &&&&&&&&&&&&&&&&&&\\
				ACTIVIDAD 2 &&&&&&&&&&&&&&&&&&\\
				ACTIVIDAD 3 &&&&&&&&&&&&&&&&&&\\
				ACTIVIDAD 4 &&&&&&&&&&&&&&&&&&\\
				ACTIVIDAD 5 &&&&&&&&&&&&&&&&&&\\
				ACTIVIDAD 6 &&&&&&&&&&&&&&&&&&\\
				ACTIVIDAD 7 &&&&&&&&&&&&&&&&&&\\
				ACTIVIDAD 8 &&&&&&&&&&&&&&&&&&\\
				ACTIVIDAD 9 &&&&&&&&&&&&&&&&&&\\
				ACTIVIDAD 10 &&&&&&&&&&&&&&&&&&\\
				ACTIVIDAD 11 &&&&&&&&&&&&&&&&&&\\
				ACTIVIDAD 12 &&&&&&&&&&&&&&&&&&\\
				ACTIVIDAD 13 &&&&&&&&&&&&&&&&&&\\
				ACTIVIDAD 14 &&&&&&&&&&&&&&&&&&\\
				ACTIVIDAD 15 &&&&&&&&&&&&&&&&&&
			\end{longtblr}
	\chapter{Resultados esperados}
	
	¡ESTO ES UNA PLANTILLA Y ESTÁ HECHA A LAS PRISAS, MODIFIQUEN LO QUE TENGAN QUE MODIFICAR Y AJUSTEN ESTE DOCUMENTO A SUS NECESIDADES!
	
	\clearpage
	\newpage	
	
	% Se imprimen las referencias
	\printbibliography[heading=bibintoc]
	
	% Se inicia la sección de apéndices
	\appendix
	
	% Se incluyen todos los apéndices
	\includepdf[%
	pages = 1,%
	pagecommand = {%
		\chapter{Especificaciones de Arduino Uno R3}%
		\label{ch:arduino-uno-r3}%
		\thispagestyle{generalfancy}%
	}%
]{appendix/A000066-datasheet.pdf}

\includepdf[%
	pages = {1,2-4,6},%
	pagecommand = {%
		\thispagestyle{generalfancy}%
	}%
]{appendix/A000066-datasheet.pdf}
\end{document}