\chapter{Planteamiento del problema}

	{\Large \textbf{Síntesis de } \cite{itriago_c_planteamiento_2011}}
	
	\textbf{	Problema:}
	
	\begin{itemize}
		\item Estoy aquí ¿Cómo puedo llegar allá?
		\item Definir punto de partida, definir punto de llegada
		\item Necesidad actual
		\item Dato, meta, condiciones y operaciones
	\end{itemize}
	
	El planteamiento del problema constituye una demostración de cómo se está percibiendo la situación que se quiere investigar o resolver, de allí su importancia para la definición del Proyecto de Investigación.
	
	\begin{itemize}
		\item Puedo presentar en un par de párrafos el problema a tratar
		\item Indicar las restricciones
		\item Deriva de una necesidad de mejora
		\item Requieren herramientas y conocimiento
		\item ¿Opitimizas o inventas?
	\end{itemize}
	
	\textbf{INTERROGANTES}
	
	\begin{itemize}
		\item ¿Qué es lo que está sucediendo y cómo está sucediendo?
		\item ¿Desde cuándo?
		\item ¿Dónde está sucediendo?
		\item ¿Quién ha investigado antes sobre eso?
		\item ¿Por qué sucede?
		\item ¿Qué resulta problematizado para lo que se describe?
		\item ¿Para quién o quienes es un problema?
		\item ¿Qué puede hacerse para intentar resolver el problema?: se formula aquí claramente la meta o el propósito para la situación planteada, indicando de manera general lo que se puede hacer o bien la forma como se puede operar para superar exitosamente el problema que se está planteando
	\end{itemize}